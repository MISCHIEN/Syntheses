
\chapter*{Introduction}

Fluid dynamics is based on continuity hypothesis, all quantities can be expressed as a continuous function of time and space coordinates. The governing equations are partial differential equations. Because of the geometrical complexity of the domain and of the equations, we need strategies. The first one is to forget about the equations and to rely on experiments. The second is to consider simplified cases, and approximate theoretical analytic solutions (aerodynamics). The third approach is numerical approach. Disadvantages and advantages of the different methods can be listed as: \\

\begin{itemize}
\item[•] \textbf{Experimental:} the advantage is that it is the most realistic, but requires equipment, there are scale problems (similarity), interferences (tests in finite space), measurement difficulties and operating costs.
\item[•] \textbf{Theoretical}: the advantage is that we have a mathematical expression and we don't have to repeat calculus, but it is restricted to simple geometries and linear problems. 
\item[•] \textbf{Numerical}: the advantages are that we can dead with complex geometries, non linear problems and unsteady problems, but there are truncation errors, problems with boundary conditions like the finite space in experiment and the computation cost. \\
\end{itemize}

In reality these approaches are complementary. We can use the second method to simplify the numerical computations, crucial for example for costly operations like computations on turbulence. The evolution of numerical cost over the past 40 years has been particularly impressive, cost decreased dramatically. In the other hand, the experimental cost tends to increase (technical personal, material,…). Nowadays we can measure many things impossible to measure before. This explains why the numerical computations have spread incredibly. \\

The design relies mainly on the numerical methods and less on the experimental testing, but it is still needed to confirm the data. We can use the numerical methods in many fields and we could call this « numerical physics ». We should deal with this in a single course of computational method and then to specialize it to the specific disciplines. An approximate solution to a problem is some kind of mathematical entity, an object, which depends on a finite number of real parameters, and which constitutes a representation of the continuous field under study. The numerical solution belongs to a finite dimension space whereas the theoretical to an infinite. \\

There are different types of numerical representation:

\begin{itemize}
\item[•] \textbf{Discrete:} collection of either point values (samples of the solutions) or subdomain averages. We are not able to give an exact solution on basis of these points, but rather an estimation.
\item[•] \textbf{Functional:} the solution is expressed as a function $u*(x) = f(x,a_i)$, depending on a set of variables. Most of the time the dependence is linear: $u* = \sum_{i=1}n a_i v_i(x)$ where $v_i(x)$ are a priori specified functions. \\
\end{itemize}

Ones we have chosen the numerical representation method, we have to generate a system of algebraic equations linking the parameters from the representation and the governing equations. The last step is to solve the system. The step of generating the equations system is called \textbf{discretization}. For one problem, several discretizations are possible. Some examples are given, consult the syllabus for more details. 