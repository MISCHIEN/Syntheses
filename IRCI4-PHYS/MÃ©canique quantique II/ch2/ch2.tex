\chapter{Systèmes de particules identiques}
\section{Origine du problème}
Imaginons que nous avons deux boules de billard en déplacement, que nous laissons évoler dans le temps. 
Une modélisation mathématique pourrait être
\begin{equation}
\left\{\begin{array}{ll}
\vec{r_1}(t) &= \vec{f}(t)\\
\vec{r_2}(t) &= \vec{g}(t)
\end{array}\right.
\end{equation}
Or, ceci n'est qu'une modélisation possible : la description reste identique si l'on permute les numéros. Ceci 
est vrai en physique classique, mais pas forcément dans le cadre de la mécanique quantique : lorsque l'on 
travaille avec des objets quantique, il faut traiter avec des paquets d'onde. Lorsque ceux-ci seront proche l'un 
de l'autre ils vont se recouvrir : les deux situations ci-dessous sont indiscernables
\begin{center}
Inclure figure cours 2, sous eq (13).
\end{center}


\subsection{Deux particules identiques sans interactions dans un O.H.}
Soit l'Hamiltonien de l'oscillateur harmonique à deux particules
\begin{equation}
\begin{array}{ll}
\hat{H} &\DS= \frac{p_1^2}{2m}+\frac{1}{2}m\omega^2\vec{r_1^2}+\frac{p_2^2}{2m}+\frac{1}{2}m\omega^2\vec{r_2^2}\vspace{2mm}\\
&= h^{(1)} + h^{(2)}
\end{array}
\end{equation}
L'énergie vaut 
\begin{equation}
E = (1+n_1+n_2)\hbar\omega,\qquad\ n_1,n_2\geq 0
\end{equation}

\subsubsection{État fondamental $E_0$}
L'énergie de l'état fondamental est donnée par $E_0=\hbar\omega$ et la fonction propre par 
\begin{equation}
\psi_0(x_1,x_2) = \phi_0(x_1)\phi_0(x_2)
\end{equation}

\subsubsection{Premier état excité $E_1$}
L'énergie et la fonction d'onde est cette fois donnée par
\begin{equation}
E_1 = 2\hbar \omega,\qquad \psi_1(x_1,x_2) = \phi_1(x_1)\phi_0(x_1)
\end{equation}
où $\phi_0$ désigne l'état fondamental. Nous pouvons avoir exactement l'opposé en permutant le rôle des deux 
particules ou encore en considérant une combinaison linéaire de ces deux situations
\begin{equation}
\begin{array}{ll}
\psi_1' &= \phi_0(x_1)\phi_1(x_2)\\
\psi_1'' &=\alpha \phi_1(x_1)\phi_0(x_2)+\beta \phi_0(x_1)\phi_1(x_2)
\end{array}
\end{equation}
Regardons ce qui se produit lorsque l'on applique l'Hamiltonien à cet état
\begin{equation}
\begin{array}{ll}
\DS(\hat{h}^{(1)}+\hat{h}^{(2)})(\alpha\phi_1\phi_0+\beta\phi_0\phi_1) &\DS= \alpha\frac{3\hbar \omega}{2}\phi_1\phi_0 + \beta
\frac{\hbar\omega}{2}\phi_0\phi_1 + \alpha\frac{\hbar\omega}{2}\phi_1\phi_0+\beta\frac{3\hbar\omega}{2}\phi_0\phi_1\vspace{2mm}\\ 
&\DS= 
2\hbar\omega(\alpha\phi_1\phi_0+\beta\phi_0\phi_1)
\end{array}
\end{equation}
Or
Nous pouvons ainsi interpreter $(\alpha\phi_1\phi_0+\beta\phi_0\phi_1)$ comme une fonction propre et $2\hbar\omega$ comme 
la fonction propre associée. Nous somme face à la \textbf{dégénérescence d'échange} qui est causée par l'échange de particules 
identiques : les situations sont indiscernables.\\

Intéressons nous à la valeur moyenne du produit des deux observables positions
\begin{equation}
\begin{array}{ll}
\langle x_1x_2\rangle_\psi &=\DS (\alpha^*\bra{\phi_1}\bra{\phi_0}+\beta^*\bra{\phi_0}\bra{\phi_1})x_1x_2
((\alpha\ket{\phi_1}\ket{\phi_0}+\beta\ket{\phi_0}\ket{\phi_1}))\\
&=|\alpha|^2\bra{\phi_1}x\ket{\phi_1}\bra{\phi_0}x\ket{\phi_0} + \alpha^*\beta \bra{\phi_1}x\ket{\phi_0}\bra{\phi_0}x\ket{\phi_1} \\&\ \ \ + \alpha\beta^*\bra{\phi_0}x\ket{\phi_1}\bra{\phi_1}x\ket{\phi_0} + |\beta|^2\bra{\phi_0}x\ket{\phi_0}\bra{\phi_1}x\ket{\phi_1}
\end{array}
\end{equation}
Or $bra{\phi_1}x\ket{\phi_1} = bra{\phi_0}x\ket{\phi_0} = 0$ : il n'y a pas de raison que la valeur moyenne soit "plus à gauche" 
ou "plus à droite (cf. cours de J.M.Sparenberg)\footnote{La fonction d'onde est symétrique ou antisymétrique, mais en module il y a autant de chance que la particule soit à gauche ou à droite.}.\\

Pour continuer le calcul, introduisons
\begin{equation}
x = \sqrt{\frac{\hbar}{m\omega}}X\quad\Leftrightarrow X = \frac{a+a^\dagger}{\sqrt{2}}\quad \Rightarrow \quad \bra{1}X\ket{0} =
\frac{1}{\sqrt{2}}\bra{1}a\ket{0} + \frac{1}{\sqrt{2}}\bra{1}a^\dagger\ket{0} = \frac{1}{\sqrt{2}}
\end{equation}
Nous avons alors
\begin{equation}
\bra{\phi_1}x\ket{\phi_0} = \sqrt{\dfrac{\hbar}{2m\omega}}
\end{equation}
Dès lors
\begin{equation}
\langle x_1x_2\rangle_\psi = \alpha^*\beta \frac{\hbar}{2m\omega}+ \alpha\beta^*\frac{\hbar}{2m\omega} = \frac{\hbar}{
2m\omega}\Re(\alpha^*\beta) \quad\Rightarrow\quad ?
\end{equation}
Le choix de $\alpha$ et $\beta$ est totalement arbitraire et la solution actuelle ne dépend en rien de l'observable 
initiale : il manque quelque chose sans quoi il n'est possible de rien prédire. Ce "quelque chose manquant" n'est rien 
d'autre que le principe d'exclusion de Pauli, nous reviendrons donc plus tard sur ce développement.

\section{Opérateur d'échange}
\subsection{Propriétés, valeurs propres, opérateur (anti)-symétriseur}
\subsection{Cas du spin $1/2$}
Soit $\ket{k}_1$ une base de $\mathcal{H}_1$, $\ket{n}_2$ une base de $\mathcal{H}_2$ et $\mathcal{H} = 
\mathcal{H}_1\times\mathcal{H}_2$. Nous pouvons écrire $\ket{\psi}$ comme
\begin{equation}
\ket{\psi} = \sum_{k,n} C_{k,n} \ket{k}_1\ket{n}_1
\end{equation}
Il faut maintenant introduire notre \textbf{opérateur d'échange}. Par définition
\begin{equation}
\hat{P}_{12}\ket{k}_1\ket{n}_2 = \ket{n}_1\ket{k}_2
\end{equation}
Appliquons cet opérateur comme suggéré ci-dessous
\begin{equation}
\hat{P}_{12} \ket{\psi}_1\ket{\phi}_2=\ket{\phi}_1\ket{\psi}_2
\end{equation}
Où encore, par décomposition dans la base de $\mathcal{H}$
\begin{equation}
\begin{array}{ll}
\DS\hat{P}_{12}\left(\sum_k \alpha_k\ket{k}_1\right)\left(\sum_n \beta_n\ket{n}_2\right) &=\DS \sum_k\sum_n \alpha_k
\beta_n \ket{n}_1\ket{k}_2\\
&=\DS \sum_n\sum_k \beta_k\ \alpha_n \ket{k}_1\ket{n}_2 = \ket{\phi}_1\ket{\psi}_2
\end{array}
\end{equation}
A partir de la définition, on peut montrer que
\begin{equation}
\hat{P}_{12}= \sum_{k,n} \ket{n}\bra{k}\ \ \otimes\ \ \ket{k}\bra{n}
\end{equation}

\subsubsection{Propriétés}
Il existe plusieurs propriétés, nous allons ici en présenter quatre 
\begin{enumerate}
\item $\hat{P}_{12}=\hat{P}_{21}$
\item $\hat{P}_{12}^2 = \hat{\mathbb{1}}$ (on fait apparaître quatre delta de Kronecker : résultat attendu car une double permutation d'indice revient à ne rien faire).
\item $\hat{P}_{12}$ est unitaire : $\hat{P}_{12}\hat{P}_{12}^{-1} = \hat{\mathbb{1}}$
\item $\hat{P}_{12}^\dagger = \hat{P}_{12} = \hat{P}_{12}^{-1}$ (voir séance d'exercices)
\end{enumerate}

\subsubsection{Valeurs propres et opérateur (anti)-symétriseur}
L'opérateur d'échange possède deux valeurs propres : $+1$ (\textit{symétrique}) et $-1$ (\textit{antisymétrique}). On 
définit alors deux opérateurs : l'opérateur symétriseur et l'opérateur anti-symétriseur
\begin{equation}
\left\{\begin{array}{ll}
\hat{S} &= \frac{1}{2}\left(1+\hat{P}_{12}\right)\\
\hat{A} &= \frac{1}{2}\left(1-\hat{P}_{12}\right)
\end{array}\right.
\end{equation}
Ces deux opérateurs vérifie les propriétés suivantes
\begin{equation}
\hat{S}^2 =\hat{S},\qquad \hat{A}^2=\hat{A},\qquad \hat{S}\hat{A}=\hat{A}\hat{S} = 0, \qquad \hat{S}+\hat{A}=\hat{\mathbb{1}}
\end{equation}
Les noms de ces opérateurs se comprennent facilement avec la propriété énoncée ci-dessous
\begin{equation}
\left\{\begin{array}{ll}
\hat{P}_{12}\hat{S} = \hat{S}\hat{P}_{12} = \hat{S}\\
\hat{P}_{12}\hat{A} = \hat{A}\hat{P}_{12} = -\hat{A}
\end{array}\right.
\end{equation}
Si on applique $\hat{P}_{12}$ sur un état, on obtient
\begin{equation}
\begin{array}{ll}
\DS\hat{P}_{12}\underbrace{\hat{S}\ket{\psi}}_{\ket{\zeta}}&\DS=\underbrace{\hat{S}\ket{\psi}}_{\ket{\zeta}}\\
\DS\hat{P}_{12}\underbrace{\hat{A}\ket{\psi}}_{\ket{\xi}}&\DS=-\underbrace{\hat{A}\ket{\psi}}_{\ket{\xi}}
\end{array}
\end{equation}
où $\ket{\zeta}\subset$ sous-espace symétrique et $\ket{\xi}\subset$ sous-espace anti-symétrique.


\subsection{Cas du spin $1/2$}
Avant d'introduire le spin de particules identiques, rappellons ce qui a été vu : l'opérateur d'échange
\begin{equation}
\hat{P}_{12} = \sum_{n,k} \ket{k}\bra{n}\otimes\ket{n}\bra{k}
\end{equation}
Pour deux particules sans spin, nous avions que $\hat{P}_{12}\psi(\vec{r_1},\vec{r_2})=\psi(\vec{r_2},\vec{r_1})$, 
c'est-à-dire
\begin{equation}
\begin{array}{ll}
\ket{\psi} &=\DS \iint d\vec{r}_1d\vec{r}_2\ \psi(\vec{r_1},\vec{r_2})\ket{\vec{r_1}}\bra{\vec{r_2}}\vspace{2mm}\\
\hat{P}_{12}\ket{\psi}&=\DS \iint d\vec{r}_1d\vec{r}_2\ \psi(\vec{r_1},\vec{r_2})\ket{\vec{r_2}}\bra{\vec{r_1}}\vspace{2mm}\\
&=\DS \iint d\vec{r}_1d\vec{r}_2\ \psi(\vec{r_2},\vec{r_1})\ket{\vec{r_1}}\bra{\vec{r_2}}
\end{array}
\end{equation}
Si nous traitons maintenant deux particules identiques possédant un spin et que l'on applique à ce système $\hat{P}_{12}$, 
le spin doit aussi changer. On définit alors 
\begin{equation}
\hat{P}_{12} = \hat{P}_{12}^{\text{(spatial)}}\times\hat{P}_{12}^{\text{(spin)}}
\end{equation}
En toute généralité, nous pouvons écrire\footnote{l'equation est fausse il manque un <m | et un |r> dans l'integrale }\footnote{Relation de fermeture et ? }
\begin{equation}
\ket{\psi} = \int d\vec{r}\underbrace{\sum_m\ket{m}\overbrace{\bra{\vec{r}}m\ket{\psi}}^{\psi_m(\vec{r})}}_{\ket{\psi(\vec{r})}}
\end{equation}
En appliquant l'opérateur d'échange
\begin{equation}
\begin{array}{ll}
\hat{P}_{12} &=\DS \hat{P}_{12} \sum_{m_1}\sum_{m_2} \psi_{m_1,m_2}(\vec{r}_1,\vec{r_2})\ket{m_1}\ket{m_2}\vspace{2mm}\\
&=\DS \hat{P}_{12} \sum_{m_1}\sum_{m_2} \psi_{m_1,m_2}(\vec{r}_1,\vec{r_2})\ket{m_2}\ket{m_1}\vspace{2mm}\\
&=\DS \hat{P}_{12} \sum_{m_1}\sum_{m_2} \psi_{m_2,m_1}(\vec{r}_2,\vec{r_1})\ket{m_1}\ket{m_2}
\end{array}
\end{equation}
Lorsque l'on applique un tel opérateur, il est important de noter que "tout" s'échange.

\section{Symétrie des états quantiques}
\subsection{Cas de $2\to N$ particules identiques, principe de Pauli (lien spin-statistique)}
Venons-en au principe de Pauli qui va nous permettre de résoudre le problème de deux particules 
identiques.  Soit
\begin{equation}
\hat{P}_{12}\ket{m_1=\pm1/2}\ket{m_2=\pm1/2} = \ket{m_2}\ket{m_1}\qquad\text{une base couplée}
\end{equation}
Nous avons donc comme base couplée
\begin{equation}
S^2,\qquad S_z,\qquad S = S_1+S_2
\end{equation}
de sorte que $s=0$ ou $s=1$. Compte-tenu de ceci, on peut définir l'état \textbf{triplet} ($s=1$)
\begin{equation}
\left\{\begin{array}{ll}
\ket{s=1,m=+1} &= \ket{m_1=1/2}\ket{m_2=1/2}\\
\ket{s=1,m=0} &= \frac{1}{\sqrt{2}}\ket{1/2}\ket{-1/2}+\ket{-1/2}\ket{1/2}\\
\ket{s=1,m=-1} &= \ket{-1/2}\ket{-1/2}
\end{array}\right.\quad \Rightarrow \hat{P}_{12}\ket{s=1,m} = \ket{s=1,m}
\end{equation}
où nous avons utilisé les coefficients de Clebsch-Gordan. Cet état correspond à la situation ou 
les spins sont alignés de sorte que cet état soit symétrique à l'échange. Nous avons d'autre 
part l'état \textbf{singlet} $(s=0)$, antisymétrique à l'échange
\begin{equation}
\left\{ \ket{s=0, m=0} \frac{1}{\sqrt{2}}\left(\ket{1/2}\ket{1/2} - \ket{-1/2}\ket{-1/2}\right)\right.
\quad \Rightarrow \hat{P}_{12}\ket{s=0,m=0} = -\ket{s=0,m=0}
\end{equation}


Afin de montrer clairement que le nom des particules n'a pas de sens, définissions un opérateur 
d'échange et appliquons-le
\begin{equation}
\hat{P}_{12} \ket{\psi} = e^{i\delta}\ket{\psi}\qquad\forall\psi
\end{equation}
où nous savons que la phase globale $e^{i\delta}$ est irrelevante. Or comme $\hat{P}_{12}^2=
\hat{\mathbb{1}}$, nous devons avoir
\begin{equation}
\hat{P}_{12}^2 \ket{\psi} = e^{2i\delta}\ket{\psi}=\ket{\psi}\qquad\forall\psi
\end{equation}
Dès lors, il faut que
\begin{equation}
e^{2i\delta} = 1\qquad\Leftrightarrow\qquad e^{i\delta}=\pm1
\end{equation}
On peut alors écrire
\begin{equation}
\hat{P}_{12}\ket{\psi} = \pm\ket{\psi}
\end{equation}
Lorsque l'on applique un opérateur d'échange, le résultat doit être symétrique ou antisymétrique. 
Considérons un état $\ket{\psi}$
\begin{equation}
\left.\begin{array}{ll}
\ket{\psi} &=\DS \sum_{k,n}C_{k,n}\ket{k}\ket{n}\\
\hat{P}_{12}\ket{\psi} &=\DS \sum_{k,n}C_{n,k}\ket{k}\ket{n}
\end{array}\right\}\quad\Rightarrow C_{k,n} = \pm C_{n,k}\quad \forall k,n
\end{equation}
On peut alors écrire la relation de proportionnalité suivante\footnote{Un état pour la particule 1 et un
second pour la particule 2.} (on ne s'occupe pas ici de la normalisation)
\begin{equation}
\begin{array}{ll}
\ket{\psi_S} &\propto\DS \sum_{k,n} d_{k,n}\left(\ket{k}\ket{n}+\ket{n}\ket{k}\right)\\
\ket{\psi_A} &\propto\DS \sum_{n>k} d_{k,n}\left(\ket{k}\ket{n}-\ket{n}\ket{k}\right)
\end{array}
\end{equation}
L’application d'un interchangement entre ces deux systèmes doit provoquer l'apparition d'un 
signe positif ou négatif. En toute généralité
\begin{equation}
\psi(x_1,x_2) = \phi_1(x_1)\phi_0(x_2) + \beta \phi_0(x_1)\phi_1(x_2)
\end{equation} 
L'échange des deux particules fera donc apparaître un signe positif ou négatif en fonction de si 
la fonction est symétrique ou antisymétrique.\\

La connaissance de ceci nous permet de résoudre le problème précédemment posé, à savoir le calcul 
de la valeur moyenne du produit de deux positions. Nous avions trouvé
\begin{equation}
\langle x_1x_2\rangle = \frac{\hbar}{m\omega}\Re(\alpha^*\beta)
\end{equation}
Nous avons donc
\begin{equation}
\begin{array}{lll}
\DS\to \psi_S : \alpha = \beta &\DS= \frac{1}{\sqrt{2}} &\DS\Rightarrow \langle x_1x_2\rangle = \frac{\hbar}{2m\omega}\vspace{2mm}\\
\DS\to \psi_A : \alpha = -\beta &\DS= \frac{1}{\sqrt{2}} &\DS\Rightarrow \langle x_1x_2\rangle = -\frac{\hbar}{2m\omega}
\end{array}
\end{equation}

Ceci est une bonne opportunité d'énoncer le \textsc{Principe de Pauli}:\\

\textit{Toute particule doit être un boson ou un fermion et si les particules identiques sont des }
 \begin{equation}
 \begin{array}{ll}
 \text{Boson } &\to \ket{\psi} \text{ doit être symétrique sous } \hat{P}_{12}\\
 \text{Fermion } &\to \ket{\psi} \text{ doit être anti-symétrique sous } \hat{P}_{12}
 \end{array}
 \end{equation}
 
 Ce principe nous informe aussi qu'être boson ou fermion ne dépend que du spin : un boson possède un 
 spin entier (par exemple, le photon) alors qu'un fermion est muni d'un spin demi-entier (par exemple, 
 l'électron).\\
 
 Il est bon de savoir que le spin peut être complètement décrit à l'aide du "Spin Statistics Theorem" qui 
 sort du cadre de ce cours. Pour synthétiser :
\begin{equation}
\begin{array}{ll}
\psi_S \to \text{Boson}\\
\psi_A \to \text{Fermion}
\end{array}
\end{equation}
Il est souvent "dit" que les bosons peuvent rester ensemble alors que les fermions doivent être "séparés. Ceci 
peut se deviner, $\langle x_1 x_2\rangle$ n'est en réalité qu'un coefficient de corrélation entre les deux 
particules. Si celui-ci est nul, il n'y a pas de corrélation entre les particules (cas classique). Ici, à cause 
du principe de Pauli nous aurons une corrélation positive ou négative.\\

Pour le cas des bosons, si l'un d'eux se trouve d'un côté, il y a de forte probabilité que le second se retrouve 
du même côté (ils occupent le même espace). Par contre, pour les fermions, si l'un est "à droite" le second 
sera plus probablement "à gauche". On parle de \textbf{bunching effects}  pour les bosons qui s'attirent et de 
\textit{repulsion} pour les fermions qui se repoussent. Cette répulsion vient du fait que le vecteur d'onde est 
ansitymétrique. Cette répulsion est due au principe de Pauli et \textbf{pas} un phénomène physique\footnote{Notons 
que pour le calcul des impulsions, on retrouvera le même signe positif pour les bosons et le signe négatif pour 
les fermions.}. Notons également que la symétrie de la fonction d'onde joue un rôle crucial en spectroscopie.



\subsubsection{Deux particules identiques sans spin}
Ayant un spin nul, ils 'agit forcément de bosons : la fonction d'onde sera symétrique et sa modélisation sera simple
\begin{equation}
\psi(\vec{r_1},\vec{r_2}) =\psi(\vec{r_2},\vec{r_1}) 
\end{equation}

 
\subsubsection{Deux particules de spin $1/2$}
Considérons par exemple deux électrons. Nous savons que l'opérateur d'échange agi sur la partie spatiale et le spin. Dès 
lors, plusieurs cas sont possible pour la partie spatiale en fonction de l'état singlet et triplet, la fonction d'onde devant 
être anti-symétrique. Écrivons notre état comme une combinaison linéaire des vecteurs de base
\begin{equation}
\ket{\psi(\vec{r_1},\vec{r_2})} = \underbrace{\psi_{0,0}(\vec{r_1},\vec{r_2})}_{(S)}\underbrace{\ket{s=0,m=0}}_{\text{singlet } (A)} + \underbrace{\sum_{m=-1}^{+1} \psi_{1,m}(\vec{r_1},\vec{r_2})}_{(A)}\underbrace{\ket{s=1,m}}_{\text{triplet } (S)}
\end{equation}
Ceci "vit" dans un espace quadridimensionnel ($2D\times2D$) où nous utilisons une base couplée comme une somme entre l'état 
singlet et triplet. Comme nous avons deux fermions, il faut bien que la fonction totale soit anti-symétrique : si la partie 
"spin" est symétrique, la partie "spatiale" doit être anti-symétrique et inversement.
\begin{equation}
\left\{\begin{array}{lll}
\psi_{0,0}(\vec{r_1},\vec{r_2}) &=\psi_{0,0}(\vec{r_1},\vec{r_2})&\quad \forall \vec{r_1},\vec{r_2}\\
\psi_{1,m}(\vec{r_1},\vec{r_2}) &=-\psi_{1,m}(\vec{r_1},\vec{r_2})&\quad \forall \vec{r_1},\vec{r_2},\ m=0,\pm1
\end{array}\right.
\end{equation}
Ceci a des conséquences importantes en spectroscopie.

\subsubsection{Interaction d'échange}
Considérons l'atome d'hélium $He$
\begin{equation}
\hat{H} = \frac{p_1^2}{2m}+\frac{p^2_2}{2m} - \frac{2e_1^2}{r_1}- \frac{2e_1^2}{r_2}+- \frac{e_1^2}{r_{12}}
\end{equation}
Si on regarde le spectre on montre que l'état fondamental est singlet (symétrie spatiale) alors que le premier état excité est triplet (antisymétrique spatialement).\\

Si l'on regarde le gap entre l'état fondamental et le premier état excité, celui-ci vaut $20\ eV$. On pourrait penser 
que cette grande différence soit du à une interaction d'origine magnétique : les spin ont intérêt d'être anti-aligné. 
Il existe bien une interaction d'origine magnétique due au spin, mais celle ci est $\approx 0.2\ eV$. Cette levée de 
dégénérescence trouve une interprétation avec le principe de Pauli : si la partie spatiale est symétrique, les deux 
électrons vont se pouvoir occuper le même état et occuper le fond du puits. Par contre, si la partie spatiale est 
antisymétrique les particules sont des fermions : les spins étant identiques, ils ne peuvent occuper la même orbitale 
et un électron devra aller dans une orbitale "plus haute". Ce gap correspond donc à la différence d'énergie entre les 
deux orbitales qui est d'origine coulombienne et non magnétique.

\subsubsection{Principe d'exclusion de Pauli pour des fermions indépendants}
Nous avons dans ce cas
\begin{equation}
\hat{H} = \hat{h_1}+\hat{h_2},\qquad \hat{h}\ket{n} = \mathcal{E}\ket{n}
\end{equation}
Un candidat de base non-couplée est par exemple $\ket{n}_1\ket{n'}_2$. Si $n=n'$ ça va pas, c'est exclu par le 
principe de Pauli : occuper la même orbital est interdit. Par contre\footnote{C'est quoi?}
\begin{equation}
n\neq n' : \frac{1}{\sqrt{2}}(\ket{n}_1\ket{n'}_2-\ket{n'}_1\ket{n}_2)
\end{equation}
On sait pas nommer les particules et les distinguer, on peut pas les "colorer". Deux fermions ne peuvent pas occuper le 
même état mais pour les fermions indépendant, chacun occupe "son" espace\footnote{A éclaircir, lien de l'indépendance non clair.} 


\subsection{Groupe symétrique $S_N$, opérateur de permutation et (anti)-symétriseur}
Lorsque nous sommes face à un système composé de $N$ particules identiques, au lieu de considérer $\hat{P}_{12}$ 
on généralise avec $\hat{P}_{i,j}$. Le principe reste identique si ce n'est que l'on parle d'opérateur de 
permutations, toutes les permutations possibles étant envisagées
\begin{equation}
\hat{P}_{i,j} \ket{\psi} = \underbrace{e^{i\delta}}_{\pm 1}\ket{\psi}
\end{equation}

\subsection{Écriture générale pour $N$ bosons ou fermions, déterminant de Slater}
Le principe de Pauli comporte un postulat sur la symétrisation 
\begin{itemize}
\item[$\bullet$] Un vecteur d'état de $N$ bosons identiques est totalement symétrique à l'échange
\item[$\bullet$] Un vecteur d'état de $N$ Fermions identiques est totalement antisymétrique à l'échange
\end{itemize}
Il faut donc considérer que toutes les paires peuvent s'échanger (permutations). Les relations triangulaires 
nous disent que
\begin{equation}
|j_1-j_2|\leq j \leq j_1+j_2
\end{equation}
Celles-ci nous disent que la combinaison d'un nombre pair de boson (spin entier) donne un boson et inversement 
pour obtenir un fermion. Dès lors, un système composé de $N$ fermions se comporte comme un boson si $N$ est 
pair et comme un fermion sinon\footnote{Revoir}. C'est bien consistant avec les relations triangulaires : un nombre pair de spin 
demi-entier donne bien quelque chose d'entier.\\


\textsc{Exemple historique}\\
Nous allons ici montrer comment l'existence du neutron peut être déduite de la symétrie de la fonction d'onde.
Soit $^{14}N$, un boson. Dans le modèle du plum-pudding les neutrons n'intervenait pas. Nous avions $A=14, Z=7$. Le
noyau était considéré de 14 protons et 7 électrons, ce qui donne 21 fermions : contradiction. Si l'on utilise le 
fait qu'un noyau est constitué de neutrons et protons, nous avons $A=14,Z=7$ et $N=7$. Le noyau est constitué de 
7 protons et 7 fermions, soit 14 fermions ce qui se comporte bien comme un boson, la contradiction est levée.\\


\subsection{Groupe symétrique : système de $N!$ permutations}
\subsubsection{1. $N$ bosons identiques}
Considérons un système de $N$ particules, toutes bosoniques
\begin{equation}
\left\{\begin{array}{ll}
1 &\to p(1)\\
2 &\to p(2)\\
\vdots\\
N &\to p(N)
\end{array}\right.,\qquad \mathcal{E}_p = (-1)^p = \pm1
\end{equation}
Il ne devrait pas être possible de les différencier. Adoptons la notation
\begin{equation}
\{\ket{i}\}
\end{equation}
qui désigne l'état d'une particule (d'une orbitale). L'état de référence est donné par
\begin{equation}
\ket{i_1}_1\ket{i_2}_2\dots\ket{i_N}_N
\end{equation}
Travaillant avec des bosons, il faut avoir quelque chose de complètement symétrique en considérant toutes les 
permutations $(N!)$
\begin{equation}
\ket{\psi_{N,boson}} = C\frac{1}{\sqrt{N!}}\sum_p\ket{i_{p(1)}}_1\ket{i_{p(2)}}_2\dots\ket{i_{p(N)}}_N
\end{equation}
Toutes ces particules sont pondérées de la même sorte : si l'on change la numérotation des particules, cela va 
juste changer l'ordre des termes de la somme mais la fonction reste symétrique :
\begin{equation}
\psi(\vec{r_1},\vec{r_2},\dots,\vec{r_N}) = C\frac{1}{\sqrt{N!}}\sum_p \phi_{p(1)}(\vec{r_1})\phi_{p(2)}(\vec{r_2})\dots 
\phi_{p(N)}(\vec{r_n})
\end{equation}
Les bosons peuvent occuper la même orbitale (on pourrait dire $i_3=i_2$ pour deux particules qui occupent le même état, 
ce n'est pas interdit pour les bosons). Dès lors, certains termes de la sommes sont identiques, les bosons pouvant 
occuper la même orbitale. Ceci est "corrigé" par le coefficient $C$.

\subsubsection{2. $N$ fermions identiques}
Similairement, pour les fermions :
\begin{equation}
\ket{\psi_N,fermion} = \frac{1}{\sqrt{N!}}\sum (-1)^p\ket{i_{p(1)}}_1\ket{i_{p(2)}}_2\dots\ket{i_{p(N)}}_N
\end{equation}
ce qui est bien anti-symétrique. Il existe une autre forme d'écrire ceci, sous la forme d'un determinant de Slater
\begin{equation}
\ket{\psi_N,fermion} = \frac{1}{\sqrt{N!}}\left|\begin{array}{cccc}
\ket{i_1}_1&\ket{i_2}_1&\dots&\ket{i_N}_1\\
\ket{i_1}_2&\ket{i_2}_2&\dots&\ket{i_N}_2\\
\vdots&\vdots&\ddots&\vdots\\
\ket{i_1}_N&\ket{i_2}_N&\dots&\ket{i_N}_N
\end{array}\right|
\end{equation}
Les termes de la diagonale correspondent à l'état de référence. Les lignes désignent les différentes particules et les 
colonnes les différents états. Le principe de Pauli s'applique directement car si deux fermions occupent le même niveau, 
deux colonnes sont identiques et le déterminent devient nul. Il est possible d'utiliser la même matrice pour décrire les 
bosons mais il faut prendre le "permanent" à la place du "déterminent" (similaire au déterminant mais où tous les signes 
négatifs deviennent positifs).

\subsection{Opérateur de permutation $\hat P$}
Si je l'applique sur cet état de référence
\begin{equation}
\hat{P}\left(\ket{i_1}_1\ket{i_2}_2\dots\ket{i_N}_N\right) = \ket{i_{p(1)}}_1\ket{i_{p(2)}}_2\dots\ket{i_{p(N)}}_N
\end{equation}
On en tire que
\begin{equation}
\begin{array}{ll}
\ket{\psi_{Boson}} &\DS\propto \frac{1}{\sqrt{N!}}\sum \hat{P}\ket{i_1}_1\ket{i_2}_2\dots\ket{i_N}_N\\
\ket{\psi_{Fermion}} &\DS\propto \frac{1}{\sqrt{N!}}\sum(-1)^p \hat{P}\ket{i_1}_1\ket{i_2}_2\dots\ket{i_N}_N
\end{array}
\end{equation}
On définit alors l'opérateur symétriseur (encadré)
\begin{equation}
\hat{S}_\pm = \frac{1}{\sqrt{N!}}(-1)^p \hat{p}
\end{equation}
où $\hat{S}_+$ est l'opérateur de symétrisation et $\hat{S}_-$ est l'opérateur d'anti-symétrisation. La raison de leur 
appellation vient de
\begin{equation}
\begin{array}{ll}
\ket{\psi_{Boson}} &= \hat{S_+} \ket{i_1}_1\ket{i_2}_2\dots\ket{i_N}_N\\
\ket{\psi_{Fermion}} &= \hat{S_-} \ket{i_1}_1\ket{i_2}_2\dots\ket{i_N}_N
\end{array}
\end{equation}


Revenons à notre opérateur (anti)-symétrisation
\begin{equation}
INCLURE DEF. 1e eq cours 4
\end{equation}
Passons en revue les différentes propriétés de cet opérateur

\begin{itemize}
\item[$\bullet$] Appliquer l'opérateur dans le \textit{bra} et le \textit{ket} du même côté ne change rien
\begin{equation}
\bra{\hat{P}\psi}\ket{\hat{P}\phi} = \bra{\psi}\ket{\phi}
\end{equation}
\item[$\bullet$] L'opérateur est hermitien. Considérons ces deux expressions équivalentes
\begin{equation}
\begin{array}{lll}
\bra{\phi}\ket{\hat{P}\psi} &= \bra{\phi}\hat{P}\ket{\psi}  &= \bra{\psi}\hat{P}^\dagger\ket{\phi}^* = 
\bra{\psi}\ket{\hat{P}^\dagger\phi}^* = \bra{\hat{P}^\dagger \phi}\ket{\psi}\\
\bra{\phi}\ket{\hat{P}\psi} &= \bra{\hat{P}^{-1}\phi}\ket{\hat{P}^{-1}\hat{P}\psi} &= \bra{\hat{P}^{-1}\phi}\ket{\psi}
\end{array}
\end{equation}
On en tire que
\begin{equation}
\hat{P}\hat{P}^{-1} = \hat{P}^{-1}\hat{P} = \hat{\mathbb{1}}
\end{equation}
\item[$\bullet$] Pour toute observable symétrique $\hat{S}$, $[\hat{P},\hat{S}]=0$. Dès lors
\begin{equation}
[\hat{P},\hat{S}] = 0\quad\Rightarrow\quad \bra{\hat{P}\psi}\hat{S}\ket{\hat{P}\phi} = 
\bra{\psi}\hat{P}^\dagger\hat{S}\hat{P}\ket{\phi} = \bra{\psi}\hat{S}\ket{\phi}
\end{equation}
car $\hat{S}\hat{P}=\hat{P}\hat{S}$ et $\hat{P}^{-1}\hat{P} = \hat{\mathbb{1}}$. Ceci se montre facilement\footnote{Soit O une observable quelquonque}
\begin{equation}
\left.\begin{array}{lll}
\bra{\hat{P}\psi}\hat{0}\ket{\hat{P}\phi} &= \bra{\psi}\hat{0}\ket{\phi}&\quad \forall\psi,\phi\\
\bra{\hat{P}^\dagger\psi\hat{P}}\hat{0}\ket{\hat{P}\phi} &= \bra{\psi}\hat{0}\ket{\phi}&\quad \forall\psi,\phi
\end{array}\right\}\quad\Rightarrow\quad \begin{array}{ll}
\hat{P}^\dagger\hat{0}\hat{P} &= \hat{0}\\
\hat{0}\hat{P} &= \hat{P}\hat{0}
\end{array}\quad\rightarrow[\hat{0},\hat{P}] = 0
\end{equation}
Un exemple simple est celui de l'hamiltonien de $N$ particules indépendantes. 
\begin{equation}
\hat{H} = \sum_{n=1}^N \hat{h}^{(i)}
\end{equation}
où toutes les permutations seront les mêmes
\end{itemize}

Dans les deux prochaines sous-sections, nous allons nous intéresser à l'état fondamental de 
$N$ particules indépendantes, c'est-à-dire que l'Hamiltonien peut (par exemple) s'écrire
\begin{equation}
\hat{H} = \sum_{n=1}^N \hat{h}^{(i)}
\end{equation}
ce qui ne comporte clairement pas de couplage (\textit{weak coupling}) entre les particules. 
Commençons par nous intéresser au cas des des bosons.

\subsection{$N$ bosons indépendants (condensats de Bose Einstein)}
Soit notre Hamiltonien 
\begin{equation}
\hat{H} = \sum_{n=1}^N \hat{h}^{(i)}
\end{equation}
avec
\begin{equation}
\hat{h}\ket{\phi_n} = e_n\ket{\phi_n}
\end{equation}
Dans la limite du zéro de température, les particules vont toutes pouvoir occuper l'état d'énergie 
le plus bas (autorisé de par leur caractère bosonique). L'énergie correspondant sera donc $N$ fois 
celle de l'état fondamental
\begin{equation}
E_0 = Ne_0
\end{equation}
Comme $N$ est généralement très grand (de l'ordre du nombre d'Avogadro), ceci devient vite une quantité
macroscopique et forme ce que l'on appelle un \textit{condensat de Bose-Einstein}.\\

Dans un tel état, toutes les particules sont si proches les unes des autres que leurs fonctions d'onde 
vont se recouvrir pour donner naissance à une fonction d'onde globale pour toutes les particules. La condition 
à respecter est 
\begin{equation}
\rho\Lambda_+^3 \geq 1
\end{equation}
où $\Lambda_+$ est la longueur d'onde de de Broglie associée à l'énergie thermique.  En effet, à très basse (et 
très  haute) température, on observe un recouvrement des longueurs de de Broglie. Ce système est assez exotique au 
niveau de sa création. Il est en effet très difficile d'atteindre des température très proche du zéro absolu.







\subsection{$N$ fermions indépendants (gaz de Fermi)}
Lorsque l'on travaille avec $N$ fermions, l'état total se doit d'être antisymétrique (principe d'exclusion 
de Pauli). Ils vont alors occuper les $N$ fermiers niveaux, jusqu'à l'énergie de Fermi (correspondant au 
dernier état occupé)
\begin{equation}
\sum_{i=0}^{N-1} e_i
\end{equation}
Lorsqu'on considère une température très basse, ces fermions vont former un \textit{gaz de fermi}. La question 
est alors de savoir : quelle est cette température minimale à atteindre. Il s'agit d'une théorie entière dont 
nous n'allons ici qu'aborder les aspects principaux.\\

Considérons une boîte $3D$ de taille $L$ dans laquelle nous allons compter le nombre de niveaux d'énergie 
présentes. Les "compter" étant difficile, on va approximer que la somme sur les particules jusqu'au niveau 
de Fermi veut à peu près le nombre d'énergie. Dans un formulaire, on peut trouver que
\begin{equation}
\rho \approx \dfrac{(2s+1)}{6\pi^2}\left(\dfrac{\hat{p_F}}{\hbar}\right)^3
\end{equation}
où $s$ est le spin et $\hat{p_F}$ l'impulsion de Pauli. L'énergie est elle donnée par
\begin{equation}
\varepsilon_F = \dfrac{\hat{p_F}^2}{2m}=\dfrac{\hbar^2}{2m}\left(\dfrac{6\pi^2\rho}{2s+1}\right)^{2/3}
\end{equation}
Celle-ci grandit comme une fonction de puissance $2/3$ de la densité. Si on calcule l'énergie de Fermi pour
un métal, on trouve
\begin{equation}
\varepsilon_F = \dfrac{\hbar^2}{2m_e}\left(\dfrac{6\pi^2\rho}{2}\right)^{2/3} \approx 2\ -\ 3\ eV
\end{equation}
où $s=\frac{1}{2}$, nous travaillons avec le gaz électronique des électrons.  L'énergie thermique à température 
ambiante est donnée par $\varepsilon_{th}= k_BT_{room}$ est vaut $0.025\ eV$. Ainsi, un gaz d'électron dans 
un métal est une bonne approximation d'un gaz de Fermi : on pourra utiliser cette théorie en physique du solide, 
pour les semi-conducteurs,\dots Ceci trouve aussi des applications en astrophysique : on peut voir une naine 
blanche comme un gros gaz de Fermi : l'énergie de Fermi correspond à une certaine pression qui empêche l'étoile
de s'effondrer sur elle même (à cause d'une immense densité) et lui confère donc une certaine stabilité.




\subsection{Émission stimulée, effet laser}
Pour conclure le chapitre, nous allons montrer que le fait de travailler avec des bosons est responsable 
de l'émission stimulée : la propriété symétrique de ceux-ci est donc fondamentale à cette application.\\

Nous n'allons ici pas faire une vraie modélisation des lasers, mais établir un modèle qui, bien qu'assez 
simple, montre la nécessite de travailler avec des bosons. Soit l'Hamiltonien et la fonction perturbation totale
\begin{equation}
\hat{H} = \sum_{i=1}^N \hat{h}^{(i)},\qquad\qquad\hat{V} = \sum_{i=1}^N v^{(i)}
\end{equation}
Nous allons lui appliquer une perturbation temporelle. Mais avant, rappelons-nous la règle d'or de Fermi
\begin{equation}
\mathbb{P}_{trans} \propto |\bra{initial}\hat{V}\ket{final}|^2
\end{equation}
La probabilité est donc liée à l'élément de matrice.  Deux  choses à considérer
\begin{enumerate}
\item On s'intéresse à la probabilité d'avoir une transition d'une particule vers un état excité
\begin{equation}
\mathbb{P}_{trans} \propto |\bra{\phi_k}\hat{v}\ket{\phi_l}|^2
\end{equation}
où l'on utilise ici $\hat{v}$, n'ayant qu'une particule.

\item On s'intéresse au cas où $N$ bosons occupent déjà l'état final : nous avons donc un total de 
$N+1$ bosons ($1$ dans l'état initial et $N$ dans l'état final).
\end{enumerate}
On peut ainsi écrire l'état initial de notre système
\begin{equation}
\begin{array}{ll}
\ket{\psi_{initial}} = \dfrac{1}{\sqrt{N+1}} &\phantom{+} \ket{\phi_k}_1\ket{\phi_l}_2\dots\ket{\phi_l}_{N+1}\\
&+ \ket{\phi_l}_1\ket{\phi_k}_2\dots\ket{\phi_l}_{N+1}\\
&\quad\qquad\qquad\dots\\
&+ \ket{\phi_l}_1\ket{\phi_l}_2\dots\ket{\phi_k}_{N+1}
\end{array}
\end{equation}
Tous les cas possibles sont ainsi pris en compte (on remarque que la diagonale n'est formé que de l'état initial).
Pour l'état final de notre système, tout est dans le niveau excité
\begin{equation}
\ket{\psi_{final}} = \ket{\phi_l}_1\ket{\phi_l}_2\dots\ket{\phi_l}_{N+1}
\end{equation}
Calculons la probabilité de transition grâce à la règle de Fermi. Calculons pour ça l'élément de matrice $\bra{initial}\hat{V}\ket{final}$ mais seulement pour un élément
\begin{equation}
\bra{\phi_k}\underbrace{\bra{\phi_l}\dots\bra{\phi_l}}_{N}\hat{v}\underbrace{\ket{\psi_l}\ket{\psi_l}\dots \ket{\psi_l}}_{N+1} = 
\dfrac{1}{\sqrt{N+1}}\bra{\phi_k}\hat{v}\ket{\psi_l}
\end{equation}
Il faut pondérer cette probabilité par le fait que nous avons un total de $N+1$ états au total. Nous avons donc 
\begin{equation}
\bra{initial}\hat{V}\ket{final} = (N+1)\dfrac{1}{\sqrt{N+1}}\bra{\phi_k}\hat{v}\ket{\psi_l} = \sqrt{N+1}\bra{\phi_k}\hat{v}\ket{\psi_l}
\end{equation}
Où encore
\begin{equation}
\bra{initial}\hat{V}\ket{final} = (N+1)|v_{kl}|^2
\end{equation}
La probabilité de transition est donc proportionnelle à
\begin{equation}
\mathbb{P}_{trans} \propto \underbrace{|v_{kl}|^2}_{(*)}+\underbrace{N|v_{kl}|^2}_{(**)}
\end{equation}
où $(*)$ correspond à de l'émission spontanée et $(**)$ à de l'émission stimulée : le fait de travailler avec 
des bosons est responsable de cette émission stimulée, il est donc obligatoire de travailler avec ces-derniers.


